
% File acl2010.tex
%
% Contact  jshin@csie.ncnu.edu.tw or pkoehn@inf.ed.ac.uk
%%
%% Based on the style files for ACL-IJCNLP-2009, which were, in turn,
%% based on the style files for EACL-2009 and IJCNLP-2008...

%% Based on the style files for EACL 2006 by 
%%e.agirre@ehu.es or Sergi.Balari@uab.es
%% and that of ACL 08 by Joakim Nivre and Noah Smith

\documentclass{letter}[20pt]

\makeatletter
\let\@texttop\relax
\makeatother
% \usepackage{acl2010}
% \usepackage{times}
% \usepackage{url}
% \usepackage{amsmath}
%\setlength\titlebox{6.5cm}    % You can expand the title box if you
% really have to
% \usepackage[brazilian]{babel}
\usepackage{hyperref}
\usepackage[brazil]{babel}
\usepackage[utf8]{inputenc}
% \usepackage{graphicx}% Include figure files
\selectlanguage{brazil}

% \usepackage{simplemargins}
% \setallmargins{3cm}

\signature{Renato Fabbri, Cris Scabello, Lourival "Cuquinha" Neto, Representantes da Sociedade Civil na Comissão de Seleção dos Projetos Inscritos no Prêmio Pontos de Mídia Livre de 2010}
\address{Brasília-DF; Campinas, São Carlos, São Paulo-SP}



\Large
\begin{document}


\begin{letter}{Indicações aos Participantes do Edital Pontos de Mídia Livre 2010.}

\opening{Caros Participantes do Prêmio Midias Livres 2010,}

Abaixo estão algumas indicações que entendemos pertinentes para os desenvolvimentos dos projetos inscritos e para a utilização dos recursos disponibilizados através do edital \emph{Prêmio Midias Livres 2010}. Acreditamos que o presente documento melhorará as chances de captação de recursos das iniciativas e também permitirá ao MinC atingir melhores resultados em benefício da população e da cultura. Desejamos que esta carta possa esclarecer ao maior número de pessoas possível algumas questões chave no processo atual de produção midiática.
  \vspace{1cm}
\begin{enumerate}
  \item Uso de licenças livres para os conteúdos produzidos.

    Textos, videos, músicas, fotos, depoimentos, seja qual for o conteúdo produzido, a utilização de licenças livres permite maior circulação do material e do nome dos autores principalmente por duas razões. Em primeiro lugar, a licença garante que qualquer um possa distribuir o material associado, sendo um incentivo explícito para tal. Em segundo lugar, permite o incentivo deliberado para geração de obras derivadas utilizando quaisquer partes do produto original, bastando a escolha da licença com especificações corretas. Em ambos os casos a citação dos autores é obrigatória. Assim o material e os nomes das pessoas e entidades envolvidas circulam mais e com maior facilidade e contribuem mais para a formação e entretenimento da população e para a geração de materiais subsequêntes. Iniciativas de uso e disseminação de licenças livres são incentivadas e aconselhadas por boa parte da política atual do MinC.

    {\bf Sugestões de Leitura:}
    \begin{itemize}%{labelitemi}{$\rightarrow$}
      \renewcommand{\labelitemi}{$\rightarrow$}
      \item \href{http://pt.wikipedia.org/wiki/Licen\%C3\%A7a\_livre}{Artigo da Wikipédia sobre Licenças Livres}.
      \item \href{http://commons.wikimedia.org/wiki/Commons:Marcas\_de\_direitos\_autorais}{Apanhado sobre diferentes licenças na Wikimedia Commons}.
      \item \href{http://estudiolivre.org/tiki-index.php?page=faq+direitos+autorais}{Perguntas Frequêntes sobre Diretos Autorais}.
      \item \href{http://www.goodcopybadcopy.net/}{\emph{Filme Good Copy Bad Copy}}.
    \end{itemize}


%  \vspace{1.5cm}
\vfill
  \item Uso de ferramentas livres para a produção de conteúdo.

    Embora iniciativas mais avançadas neste sentido possam tratar do uso de hardware livre e outras tecnologias, a ênfase atual está na disseminação do uso de Software Livre (SL). O uso de SL, seja para a produção de conteúdo, seja para quaisquer outros fins, é política deliberada do MinC (por exemplo através dos programas Cultura Viva e Cultura Digital). A adoção de SL permite a criação de um repertório tecnológico comum e compartilhado da Humanidade. O SL também garante o acesso aos meios de produção a parcelas da população menos favorecidas não só do Brasil. Garante também que não seremos privados do direito ao uso destes meios.

    {\bf Sugestões de leitura e consulta:}
    \begin{itemize}
      \renewcommand{\labelitemi}{$\rightarrow$}
      \item \href{http://estudiolivre.org/tiki-index.php?page=Softwares\%20de\%20\%C3\%81udio}{Lista de Softwares Livres para a produção em áudio}.
      \item \href{http://estudiolivre.org/tiki-index.php?page=Softwares\%20de\%20Gr\%C3\%A1fico}{Lista de Softwares Livres para a produção em gráfico}.
      \item \href{http://estudiolivre.org/tiki-index.php?page=Softwares\%20de\%20V\%C3\%ADdeo}{Lista de Softwares Livres para a produção em vídeo}.
      \item Exemplos de Softwares Livres para a produção em texto: 
  OpenOffice, Scribus, Lyx, etc.
      \item \href{http://pt.wikipedia.org/wiki/Linux}{Artigo sobre Linux na Wikipédia}.
    \end{itemize}
  \vspace{1cm}



  \item Uso de formatos livres.

  Existem formatos cuja especificação de funcionamento é publicamente conhecida e se encontram disponíveis para qualquer um utilizar. Estes formatos costumam ter implementações livres, aos moldes do software livre. Exemplos de tais fortmatos são: .pdf, .txt, .png, .ogg, .flac

  {\bf Sugestão de Leitura:}
  \begin{itemize}
    \renewcommand{\labelitemi}{$\rightarrow$}
    \item \href{http://www.estudiolivre.org/tiki-index.php?page=midia\%20livre\#formatos\_livres}{Lista de formatos livres no Estúdio Livre}.
  \end{itemize}
  \vspace{1cm}



  \item Inclusão de material produzido pela comunidade em geral e inclusão da comunidade nas produções da entidade/instituição/organização.

    Desta forma incentivando, aumentando e facilitando a democratização da expressão e da utilização dos espaços disponíveis.

  {\bf Sugestão de Leitura:}
  \begin{itemize}
    \renewcommand{\labelitemi}{$\rightarrow$}
    \item Procurar na internet sobre os projetos inscritos no edital Prêmios de Mídias Livres.
  \end{itemize}
  \vfill
  

  \item Integração com a rede de cultura existente.

  Integração com pessoas específicas, com os Pontos de Cultura e com 
  outras estruturas ligadas à cultura, especialmente que tratem 
  explicitamente da produção de mídias e da comunicação. 
  Participação em listas de e-mails tais como a do Estúdio Livre, do 
  Metarreciclagem e do Submidialogia em que pode-se inclusive desenvolver o entendimento sobre o uso de licenças livres para os conteúdos produzidos e do SL.

  {\bf Sugestões para leitura e exploração:}
  \begin{itemize}
    \renewcommand{\labelitemi}{$\rightarrow$}
    \item \href{http://www.iteia.org.br/}{iTEIA  - Rede Colaborativa de Cultura, Arte e Informação}.
    \item \href{https://lists.riseup.net/www/subscribe/estudiolivre/}{Lista de emails do Estúdio Livre}.
    \item \href{http://www.partidopirata.org/listas/}{Lista de emails do Partido Pirata}.
    \item \href{https://lists.riseup.net/www/subscribe/submidialogia/}{Lista de emails do Submidialogia}.
    \item \href{https://lists.riseup.net/www/subscribe/metareciclagem/}{Lista de emails do Metareciclagem}.
  \end{itemize}
  \vspace{1cm}

\end{enumerate}

\vfill
\closing{Agradecidos pelo trabalho de todos e de cada um de vocês,}
\vfill

\end{letter}
\end{document}
